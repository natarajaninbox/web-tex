\documentclass[10pt]{article}%
\usepackage[a4paper, top=2.5cm, bottom=2.5cm, left=1.75cm, right=2.25cm]%
{geometry}

\title{Nomenclature IROS 1}
\author{Veejay Karthik}
\date{January 2023}
\usepackage{amsmath,amssymb,amsfonts}
\usepackage{tabularx}
\begin{document}

\maketitle

%\section{Nomenclature}

\begin{tabularx}{1\textwidth} { 
  | >{\raggedright\arraybackslash}X 
  | >{\centering\arraybackslash}X 
  | >{\raggedleft\arraybackslash}X | }
 \hline
 \textbf{S.No} & \textbf{Variable} & \textbf{Description}\\
 \hline
1 &$\mathcal{R}$ * &Robot Position\\
\hline
2 &$(d_i,\theta_i)$ * &Range,Angle Info Pairs\\
\hline
3 &N * &Total number of data points from the sensor\\
\hline
4   &$\mathcal{S}$ * &Start\\
\hline
5   &$\mathcal{T}$ * &Target \\
\hline
6 &$\mathcal{G}_i$ *  &Candidate Gap\\
\hline
7 &$(\mathcal{G}_i^{os},\mathcal{G}_i^{cs})$ * &Sides of a gap\\
\hline
8 &$\theta_{\mathcal{G}_i}$ * &Angular distance of gap to target\\
\hline
9 &$\theta_{\mathcal{G}_{closest}}$ * &Angle corresponding to closest gap to goal\\
\hline
10 &$\mathcal{G}_{closest}$ * &Closest gap to goal\\
\hline
11   & $\mathcal{T}_j$ * & Intermediate targets\\
\hline
 12   & $R$ ** & State - Radial Distance  \\
\hline
13   & $\psi$ ** & State - Relative Bearing \\
\hline
14 & $\mathcal{W}_i$ ** &Waypoint/Reference Point\\
\hline
15   & $U(R,\psi,\{\mathcal{W}_i\})$ ** & Feedback motion plan in the space of admissible control inputs \\
\hline
16   & $v$ *** & Control - Linear Velocity\\
\hline
17   & $\omega$ *** & Control - Angular Velocity\\
\hline
18  &$\sigma$ *** & Sliding Variable\\
\hline
19 &$(R_0,\psi_0)$ *** &Initial distance and bearing of the robot relative to the waypoint\\
\hline
20 &$V$ *** &Lyapunov Function\\
\hline
21 &$L^{-}(V,V(R_0))$ *** &Invariant region induced by the control design relative to waypoint\\
\hline
22 &$\partial L^{-}$ *** &Boundary of $L^{-}(V,V(R_0))$\\
\hline
23 &$(d,i)$ **** &distance ($d$)  of a candidate waypoint in the direction referenced by range index $i$\\
\hline
24 &$[i_{min},i_{max}]$ **** &Range indices between which the range measurements should be checked if an invariant region could be constructed\\
\hline
25 & $\zeta(i)$ **** &Planner Function - Mapping from range indices ($i$) to the maximum distance at which a waypoint can be picked at that index\\
\hline
26 &$\{\mathcal{W}^c(\zeta)\}$ ***** &Set of candidate waypoints around the robot\\
\hline
27 &$\mathcal{W}_{i+1}$ ***** &Optimal waypoint picked for navigation\\
\hline
28 &$d(.,.)$ ***** &Euclidean Distance Function\\
\hline
\end{tabularx}
The variables are employed in the sections below,
\newline Preliminaries - *, Problem Formulation - **, Control Design - ***, Planner Function - ****, IPC-A - *****
\end{document}
